%==============================================================================
\documentclass[11pt,oneside,onecolumn,letterpaper]{article}
\usepackage{times}
\usepackage[paperwidth=8.5in, paperheight=11in,
top=2.5cm, bottom=2.6cm, left=2.58cm, right=2.53cm]{geometry}
%\setlength{\textheight} {9.00in}
%\setlength{\textwidth}  {6.40in}
%\setlength{\topmargin}  {-0.50in}
%%\setlength{\headheight} {0.00in}
%%\setlength{\headsep}     {0.40in}
%\setlength{\oddsidemargin}{-0.010in}
%\setlength{\evensidemargin}{-0.00in}
%==============================================================================
%\usepackage{algorithm}
\usepackage{amssymb}
\usepackage{color}
\usepackage{booktabs}
\usepackage{graphicx}
\usepackage{latexsym}
\usepackage{subfigure}
\usepackage{wrapfig}
\usepackage{amsmath}
\usepackage{amsthm}
\usepackage[hyphens]{url}
\usepackage{pifont}
\usepackage{color}
\usepackage{colortbl}
\usepackage[lined, boxed, linesnumbered]{algorithm2e}
\usepackage[square, comma, sort&compress, numbers]{natbib}

\newcounter{alg}
\newenvironment{enum-ref}{
\begin{list}%
{[\arabic{alg}]} {\usecounter{alg}
  \setlength{\leftmargin} {0.25in}
  \setlength{\labelwidth} {0.30in}
  \setlength{\rightmargin}{0.00in}
  \setlength{\topsep}     {0.00in}}
}{\end{list}}

\newenvironment{enum-number}{
\begin{list}%
{\arabic{alg})} {\usecounter{alg}
  \setlength{\leftmargin} {0.25in}
  \setlength{\labelwidth} {0.30in}
  \setlength{\rightmargin}{0.00in}
  \setlength{\topsep}     {0.00in}}
}{\end{list}}

\newenvironment{enum-nonum}{
\begin{list}%
{$\bullet$} {
  \setlength{\leftmargin} {0.25in}
  \setlength{\labelwidth} {0.30in}
  \setlength{\rightmargin}{0.00in}
  \setlength{\topsep}     {0.00in}}
}{\end{list}}

\let\chapter\section

%==============================================================================
\pagestyle{plain}
%==============================================================================

\title{Secure UAV Communications System Design}
\author{Team Cacti, University at Buffalo}
\date{}



\begin{document}
%%
%=============================================================================
\normalsize


\maketitle
%\date{}

\renewcommand{\thepage}{System Design, Cacti, UB--\arabic{page}}
\setcounter{page}{1} \normalsize
%
%\renewcommand{\baselinestretch}{1.2}
%\normalsize
%\vspace{0.1in}
%\centerline{\textbf{\Large }}
%\renewcommand{\baselinestretch}{1.0}
%\normalsize

\newcommand{\flagRollback}{\textsf{Rollback}\xspace}

\section{System Overview}

The transitions between SEDs will be encrypted.
Only the SEDs that have the correct key can decrypt the message.
All messages in between will be authenticated to check the integrity.

\subsection{Message Encryption}
To encrypt the message, asymmetric encryption (AES) is used.
Which we chose will be AES with CBC or CTR mode.
The key is called SED\_aes\_key which is unique for each SED.
Every SED stores a list of keys for the activated SEDs which will be used to encrypt or decrypt the messages.

\subsubsection{Possible Asymmetric Encrption}
If the performance is good enough, we will additionally implement asymmetric encryption, such as RSA, to encrypt the AES key.
In this design, each SED has the AES key plus a pair of a public key (SED\_pub\_key) and a private key(SED\_pri\_key).
Also, each SED stores a list of public keys of the activated SEDs.

To send a message, SED\_a will use the AES key to encrypt the message body first (ciphertext), then use the SED\_b's public key to encrypt the AES key (en\_key).
Then send the en\_key and ciphertext to SED\_b'.

To read the message, SED\_b uses its private key to decrypt the en\_key to get the AES key, then encrypt the ciphertext. 

\subsection{Message Authentication}
To authenticate, hmac with size 512 (may change) is chosen.
The key for hmac is called SED\_hmac\_key.

\subsection{Key generation and storage}
SSS stores all keys for all SEDs and it also has its own keys.

For each provisioned SEDs, it has its own private key SED\_key which will be used to encrypt the message in transitions.
To communicate with each other, SEDs will also store other SED\_keys.

When register a SED, the SSS is responsible to generate the SED\_key and update it.
The newly inserted SED will broadcast the key to other SEDs to update the key of the associated ID.

When deregistering a SED, the SED will broadcast to other SEDs first, then the SSS will destroy the SED\_key.

  \begin{itemize}
  \item Before deployment when we add the SEDs with \textbf{add\_sed}  
      command we generate secrets in for each SED. 
      \item We will use symmetric key encryption for SED to SED communication (AES).
      \item We change the  \textbf{dockerfiles/2b\_create\_sed\_secrets.Dockerfile} file to generate the keys for each pair of SEDs to communicate with each other.
      \item Store the keys in the containers \textbf{secret} directory.
      \item As every SED will have all the key pairs in the network, authenticity of each message can be verified.
      \item Along with the AES Keys for communication we also generate one key for each SED to do \textbf{hmac} on the header and message body. 
      \item Here the hmac preserves the integrity of the message header and message body
  \end{itemize}

\subsection{Secure the Broadcast}
Whenever the register or deregister happens, this SED will broadcast its status to update the keys stored in other SEDs.
Implementing the TESLA broadcast authentication protocol is in our work pipeline, too.

\subsubsection{Message format}
The original message format is eight bytes of metadata followed by the frame body.
We will change the frame body into an encrypted body with AES plus hmac of the header and encrypted body.

  \begin{itemize}
  \item In case of SED to SED communication: \textbf{header + hmac(header + AES (body)) + AES(body)}
  \item In case of incoming message from FAA transceiver we receive the message as given format: \textbf{header + body}
  \end{itemize}
  \begin{itemize}
  \item We will add a sequence number in from of the message body. We check the sequence number from the message after decryption to prevent reply attack.
  \end{itemize}

\subsection{Build Process}

  \begin{itemize}
    \item \textbf{dockerfiles/1a\_create\_sss.Dockerfile}
    \item \textbf{dockerfiles/2b\_create\_sed\_secrets.Dockerfile}: When "add a SED to the deployment", the build process will generate secrets for each SED.
    \item \textbf{dockerfiles/2c\_build\_controller.Dockerfile}
    \item \textbf{dockerfiles/2c\_build\_controller.Dockerfile}
  \end{itemize}

\subsection{Critical Functions}
controller.c: line 172


\section{Flag Protection}

\subsection{\textsf{D}}


\end{document}
%==============================================================================
