%==============================================================================
\documentclass[11pt,oneside,onecolumn,letterpaper]{article}
\usepackage{times}
\usepackage[paperwidth=8.5in, paperheight=11in,
top=2.5cm, bottom=2.6cm, left=2.58cm, right=2.53cm]{geometry}
%\setlength{\textheight} {9.00in}
%\setlength{\textwidth}  {6.40in}
%\setlength{\topmargin}  {-0.50in}
%%\setlength{\headheight} {0.00in}
%%\setlength{\headsep}     {0.40in}
%\setlength{\oddsidemargin}{-0.010in}
%\setlength{\evensidemargin}{-0.00in}
%==============================================================================
%\usepackage{algorithm}
\usepackage{amssymb}
\usepackage{color}
\usepackage{booktabs}
\usepackage{graphicx}
\usepackage{latexsym}
\usepackage{subfigure}
\usepackage{wrapfig}
\usepackage{amsmath}
\usepackage{amsthm}
\usepackage[hyphens]{url}
\usepackage{pifont}
\usepackage{color}
\usepackage{colortbl}
\usepackage[lined, boxed, linesnumbered]{algorithm2e}
\usepackage[square, comma, sort&compress, numbers]{natbib}

\newcounter{alg}
\newenvironment{enum-ref}{
\begin{list}%
{[\arabic{alg}]} {\usecounter{alg}
  \setlength{\leftmargin} {0.25in}
  \setlength{\labelwidth} {0.30in}
  \setlength{\rightmargin}{0.00in}
  \setlength{\topsep}     {0.00in}}
}{\end{list}}

\newenvironment{enum-number}{
\begin{list}%
{\arabic{alg})} {\usecounter{alg}
  \setlength{\leftmargin} {0.25in}
  \setlength{\labelwidth} {0.30in}
  \setlength{\rightmargin}{0.00in}
  \setlength{\topsep}     {0.00in}}
}{\end{list}}

\newenvironment{enum-nonum}{
\begin{list}%
{$\bullet$} {
  \setlength{\leftmargin} {0.25in}
  \setlength{\labelwidth} {0.30in}
  \setlength{\rightmargin}{0.00in}
  \setlength{\topsep}     {0.00in}}
}{\end{list}}

\let\chapter\section

%==============================================================================
\pagestyle{plain}
%==============================================================================

\title{Secure UAV Communications System Design}
\author{Team Cacti, University at Buffalo}
\date{}



\begin{document}
%%
%=============================================================================
\normalsize


\maketitle
%\date{}

\renewcommand{\thepage}{System Design, Team Cacti, University at Buffalo--\arabic{page}}
\setcounter{page}{1} \normalsize
%
%\renewcommand{\baselinestretch}{1.2}
%\normalsize
%\vspace{0.1in}
%\centerline{\textbf{\Large }}
%\renewcommand{\baselinestretch}{1.0}
%\normalsize

\newcommand{\flagRollback}{\textsf{Rollback}\xspace}

\section{Introduction}

This is the design document of team Cacti for MITRE eCTF 2021.

\subsection{Entities}

A SED (SCEWL-Enabled Devices) is a device with a SCEWL Bus installed. 
A SED is implemented in 2 parts: 1) CPU, which runs on ARM Cortex-A + Linux (Cannot be changed); 2) 
Microcontroller, which runs on ARM Cortex-M + Firmware. 
	
SED devices include: 
1) C2 SED (fixed location, unknown \verb|SCEWL_ID|); 
2) drop-zone SED ((fixed location, unknown \verb|SCEWL_ID|)); 
3) drone SED (fly to places, unknown \verb|SCEWL_ID|)

The following summarizes the entities in the system.

\begin{itemize}
	\item A SED CPU works at the application layer. UAV ID is used to identify drone SED CPU at the application layer. UAV ID is a secret like any other data fields at the application layer.
	
	\item A SED controller is identified by a \verb|SCEWL_ID|, which is 16 bits in length. Note that a UAV ID is different from a \verb|SCEWL_ID|. The UAV ID is a secret that controlled by the CPU.
	
	\item SSS (SCEWL Security Server) manages SEDs through registration
and deregistration. SSS does not communicate with any SED besides registration
and deregistration. SSS has a SCEWL ID of 1.
	
	\item FAA Transceiver is a device that allows the FAA to communicate with any SED, bypassing the secure SCEWL protocol that we design. FAA Transceiver has a SCEWL ID of 2.
\end{itemize}

\subsection{Communication Channels}

Following the OSI network model, we divide the network into 3 layers. 
Layer 1: the physical layer, which include radio and wired. There are implemented by UNIX socks, radio.py, etc.
Layer 2: a combination of the data link layer, network layer and transport layer, which is implemented at the controller. 
Our job is to provide security mechanisms at layer 2 before forwarding the message to layer 1 or layer 3.
Layer 3: the application layer, which is out of our control and implemented at the CPU. The application layer has its own checksum.

This following summarizes the communication channels in the system.

\begin{itemize}
	\item A SED CPU can send a targeted message to another SED CPU via radio. This message will go through the SED controller. This message must be encrypted and authenticated at controller, so only the targeted SED controller can decrypt the message and any tampering to the message by an attacker can be detected. 
	
	\item A SED CPU can send out a broadcast message to all other registered SED CPUs. This message will go through the SED controller. This message must be encrypted and authenticated as well. Only registered SED controller should be able to decrypt this message, and any tampering to the message by an attacker should be detected. 
	
	\item In registration and deregistration, a SED only talks with SSS in a wired and secured channel. No further protection is needed for this channel.
	
	\item SED CPU uses the FAA channel to talk with us, sending status notifications to us.
\end{itemize}

\subsection{Message Format}

The physical layer / data link layer frame has a header described in Section 4.6 of Rules.
The header has 8 bytes.
It has 2 bytes of magic number, 2 bytes of destination SCEWL ID, 2 bytes of source SCEWL ID, 2 bytes of body length in bytes.
The header cannot be encrypted, since it is used for routing.
But, it should be authenticated to prevent tampering.

\section{Attack Models}

The attackers can carry out the following attacks:

\begin{itemize}
	\item Intercept a targeted message and try to decrypt that, getting the Package Recovery flag.

	\item Intercept a broadcast message and try to decrypt that, getting the UAV ID Recovery flag.	
	
	\item Send a targeted message to any drone SED to make it drop a package, getting the Drop Package flag.		
	
	\item Replay the redirect message from FAA to make a drove SED fly above its altitude ceiling.
	
	\item Extract secret from a crashed drone SED firmware.  
	
	\item Spoof an FAA transceiver. But, the controller must pass all FAA messages without authentication.
	
	\item Launch their own spoofed SEDs onto the network, which may run malicious images on CPU and controller.
\end{itemize}

\section{Our Design}
% SSS stores all public keys ($pk_k$) of all provisioned SEDs. 
Each SED controller $k$ has a public key pair ($pk_k, sk_k$).
Each SED stores all other SEDs' $pk_k$.
ENC and DEC stands for symmetric encryption and decryption with AES.
AENC and ADEC stands for asymmetric encryption and decryption with RSA.

\subsection{Targeted Transmission}
Whenever the SED$_a$ sends a targeted message to SED $b$, it first generates an AES key $k_a$ and IV $iv_a$. 
Then, it uses AES-GCM to encrypt and authenticate the SCEWL header ($\mathcal{H}$) and body ($\mathcal{B}$), which outputs the ciphertext ($\mathcal{C}$) and tag ($\mathcal{T}$), e.g. $\mathcal{C}, \mathcal{T}$ = ENC$_{k_a, iv_a}(\mathcal{H} || \mathcal{B})$.
Then, it encrypts $k_a$ with $pk_b$, e.g. AENC$_{pk_b}(k_a)$.
Then, it sends out $\mathcal{M}$ = $\mathcal{H} || $AENC$_{pk_b}(k_a)||iv_a||\mathcal{T}||\mathcal{C}$.
A caveat here is to calculate body length before the AES operation.

Whenever SED$_b$ receives a targeted message $\mathcal{M}$, it first checks the $\mathcal{H}$ to see if the message is intended for it. If not, discard.
Then, it uses its private key $sk_b$ to decryt the received ciphertext AES key $k_a$, e.g. $k_a$ = ADEC$_{sk_b}($AENC$_{pk_b}(k_a))$.
Then, it uses $k_a, iv_a$ to decrypt and authenticate the message, e.g. $\mathcal{C'}, \mathcal{T'}$ = DEC$_{k_a, iv_a}(\mathcal{C})$.
It compares $\mathcal{T'}$ and $\mathcal{T}$.
If they do not match, discard.
Otherwise, $\mathcal{C'}=\mathcal{H}||\mathcal{B}$.


\subsection{Broadcast Transmission}

When the SED$_a$ does a broadcast, it first generates an AES key $k_a$ and IV $iv_a$. 
Then, it uses AES-GCM to encrypt and authenticate the SCEWL header ($\mathcal{H}$) and body ($\mathcal{B}$), which outputs the ciphertext ($\mathcal{C}$) and tag ($\mathcal{T}$), e.g. $\mathcal{C}, \mathcal{T}$ = ENC$_{k_a, iv_a}(\mathcal{H} || \mathcal{B})$.
Then, it signes $k_a$ with its private key $sk_a$, e.g. SIG$_{sk_a}(k_a)$.
Then, it sends out $\mathcal{M}$ = $\mathcal{H} || $SIG$_{sk_a}(k_a)||iv_a||\mathcal{T}||\mathcal{C}$.
A caveat here is to calculate body length before the AES operation.

Whenever SED$_b$ receives a targeted message $\mathcal{M}$, it first checks the $\mathcal{H}$ to see if the message is a broadcast.
Then, it uses SED$_a$'s public key $pk_b$ to verify the received ciphertext AES key $k_a$, e.g. $k_a$ = VER$_{pk_a}($SIG$_{sk_a}(k_a))$.
Upon a successful verification, it uses $k_a, iv_a$ to decrypt and authenticate the message, e.g. $\mathcal{C'}, \mathcal{T'}$ = DEC$_{k_a, iv_a}(\mathcal{C})$.
It compares $\mathcal{T'}$ and $\mathcal{T}$.
If they do not match, discard.
Otherwise, $\mathcal{C'}$ = $\mathcal{H}||\mathcal{B}$.

\subsection{Key Generation and Storage}

Below are the details Key generation for asymmetric encryption:
\begin{itemize}
  % \item[Step1.]
  \item[Step 1.] When adding SED$_k$ ($k$ refers to \verb|SCEWL_ID|) to a deployment, SSS generates a pair of RSA keys ($pk_k$, $sk_k$) for SED$_k$.
  
  We develop \textbf{create\_secrets.py} (with \verb|SCEWL_ID| as a parameter) script to generate key pair ($pk_k$ and $sk_k$) for each provisioned SED$_k$ and run the script from  \textbf{dockerfiles/2b\_create\_sed\_secrets.Dockerfile} file.
  SSS stores the keys in files as $k.pub$, $k.pri$ at its local filesystem.

  \item[Step 2.] When building controller for SED$_k$, the key pair ($pk_k$, $sk_k$) will be copied to this controller container.
  
  At the time of building the controller of SED's \textbf{dockerfiles/2c\_build\_controller.Dockerfile}  gets invoked, we change this file to copy the secrets from the SSS container to the controller container.

  \item[Step 3.] At registration stage, the SED$_k$ uses its own private key $sk_k$ to sign the registration message and send it to the SSS.
  It sends out $\mathcal{M}$ = $\mathcal{H} || $SIG$_{sk_k}(\mathcal{H})$.
  If $k$ is in our provisoned SED list, SSS will use the associated public key $pk_k$ to verify this message, e.g. $\mathcal{H'}$ = VER$_{pk_k}$(SIG$_{sk_k}(\mathcal{H})$).
  Upon a successful verification ($\mathcal{H}$ == $\mathcal{H'}$), SSS finish the registration for SED$_k$, adds $k$ to a delegted list, and sends the public keys of all other SEDs to SED$_k$.
  
  We modify the \textbf{sss.py} to verify the provioned SEDs and send out the public keys of other SED when registration.

  \item[Step 4.] The deregistration stage is similar to registration.
  SED$_k$ will delete other public keys first, then uses its own private key $sk_k$ to sign the deregistration message and send it to the SSS.
  If $k$ is in our provisoned SED list and in our delegated list, SSS will use the associated public key $pk_k$ to verify this message.
  Upon a successful verification, SSS will remove SED$_k$ from the delegated list.

  \item[Remove SED.] When removing SED$_k$, SSS will remove the key pair ($pk_k$, $sk_k$).
\end{itemize}

  \subsection{Prevent Replay Attack}
  For communication between the SEDs, a sequence number of 4 bytes is added in front of the message body. Each SED maintains a paired sequence number table in memory of other SEDs.To prevent the replay attack, receiver SED needs to cross-check between the received value from the message and the value from the table.
 
Explanation:
  \begin{itemize}
  \item Suppose there are 5 ($a, b, c, d, e$) SEDs in the system.
  \item Each SED maintains a paired sequence number table of other SEDs with the initial value of zero.
  \item  Example: At the beginning sequence table for SED $b$:

    \begin{center}
  \begin{tabular}{ |c|c| } 
   \hline
  \textbf{SED ID} & \textbf{Sequence Number} \\
 	\hline \hline
 	a & 0 \\ 
	c & 0  \\ 
 	d & 0 \\ 
 	e & 0 \\ 
	 \hline
\end{tabular}
\end{center}
Here each row represents a paired sequence number.
The sequence number of $a$ here signifies the sequence number for communication between the SED pair $(a,b)$. SED $a$ denotes the same pair as $b$ in its table.

  \item Suppose SED$_a$ communicates with the SED $b$. SED $a$ gets the 
  paired ($a$,$b$) value from its table, increments it by one, and then adds it in front of the message body. 
  \item After successful decryption of the message, SED $b$ checks the received sequence number from the message and ($b$, $a$) paired sequence number from it's table in memory.
  \item The received message is valid only if the message sequence number is greater than the sequence number from the table. For a valid message, the receiver SED $b$ copies the sequence from the message to it's table.
  \item After communication sequence table for SED $b$:
    \begin{center}
  \begin{tabular}{ |c|c| } 
   \hline
  \textbf{SED ID} & \textbf{Sequence Number} \\
 	\hline \hline
 	a & 1 \\ 
	c & 0  \\ 
 	d & 0 \\ 
 	e & 0 \\ 
	 \hline
\end{tabular}
\end{center}
\item Similarly, suppose SED $b$ twice communicates with SED $c$ and once with SED $e$. Now the table for SED $b$ should be:
   \begin{center}
  \begin{tabular}{ |c|c| } 
   \hline
  \textbf{SED ID} & \textbf{Sequence Number} \\
 	\hline \hline
 	a & 1 \\ 
	c & 2  \\ 
 	d & 0 \\ 
 	e & 1 \\ 
	 \hline
\end{tabular}
\end{center}
    \end{itemize}


\subsection{Build Process}

  \begin{itemize}
    \item \textbf{dockerfiles/1a\_create\_sss.Dockerfile}: Generate or destory keys for SEDs registration and deregistration.
    \item \textbf{dockerfiles/2b\_create\_sed\_secrets.Dockerfile}: Store the key list for actived SEDs.
    \item \textbf{dockerfiles/2c\_build\_controller.Dockerfile}: Generate any other secrets.
    \item ...
  \end{itemize}

\section{Implementation}
\subsection{Critical Functions and Files}
\begin{itemize}
  \item sss.py: check the provisoned SEDs.
  \item controller.c: add message encryption, decryption, and authentication, update keys.
  \item others are still in progress.
\end{itemize}


% \section{Flag Protection}

% \subsection{\textsf{D}}


\end{document}
%==============================================================================
